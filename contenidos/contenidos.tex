\documentclass{article}

% ----------------------- PAQUETES ---------------------- %
% Set the font (output) encodings
\usepackage[T1]{fontenc}					% encoding de la fuente
\usepackage[spanish]{babel}				% paquete para los acentos en español
\usepackage{graphicx}							% paquete para poder usar graficos
\usepackage{float}								% paquete para poder usar \begin {figure}[H]
\usepackage{xurl}					% https://stackoverflow.com/questions/4146606/wrap-url-ignores-margin-in-bibtex-using-pdflatex
\usepackage{hyperref}							% paquete para hacer hyperreferencias a links
\usepackage{fancyhdr}							% paquete para poner cosas en footer y header
\usepackage{wrapfig}							% paquete para hacer wrapfigure
\usepackage{titling}							% paquete para modificar el espaciado de los titulos
\usepackage{titlesec}							% paquete para modificar como se ven los titulos
\usepackage{todonotes}						% paquete para hacer todos
\usepackage{subfiles}							% paquete para hacer subfiles


% ------------------------------------------------------- %

% ---------------------- CONFIGURACION ------------------ %

\graphicspath{{images/}}
% ----------------- Configuracion de hyperref ----------- %
\hypersetup{								
	colorlinks=true,
	linkcolor=black,			%modo claro
	%linkcolor=white,		%modo oscuro
	filecolor=brown,		
	urlcolor=blue,
	}
% ------------------------------------------------------- %
%-------------- Formatos de los titulos --------------%
\titleformat{\section}
	{\bfseries \LARGE}
	{}
	{0em}
	{}[\titlerule]

\titleformat{\subsection}
	{\bfseries \Large}
	{}
	{0em}
	{}

\titleformat{\subsubsection}
	{\bfseries \large}
	{}
	{0em}
	{}

\titlespacing{\section} %me permite controlar el espaciado de la seccion que le indico
	{0em}
	{0em}
	{1.5em}

\titlespacing{\subsection}
{0em} %sangria
{3em} %separacion con lo que hay arriba
{0.5em} %separacion con lo que hay abajo
%----------------------------------------------------- %

%---------- headers y footers con fancyhdr ----------- %
\pagestyle{fancy}
	%limpio el estilo
	\fancyhf{}
	
	%header
	\renewcommand{\headrulewidth}{1pt} %La linea horizontal
	\lhead{Contenidos del Club de Programación}
	\chead{}
	\rhead{Instituto Tecnológico San Bonifacio}

	%footer
	\renewcommand{\footrulewidth}{1pt} %La linea horizontal
	\lfoot{}
	\cfoot{\today}
	\rfoot{}
% ----------------- FIN DE CONFIGURACION ------------- %

\title{Contenidos del Club de Programación}
\date{\today}
\author{Espacio de Innovacion Joven}
\begin{document}
\maketitle
\thispagestyle{fancy}

\begin{abstract}
	Este club tiene como objetivo formar un espacio en los que los alumnos
	puedan aprender de forma grupal conocimientos claves para el ámbito informático
	del mundo moderno.
\end{abstract}

\section{Contenidos de club}

\subsection{Conocimientos Básicos de Informática}
\begin{itemize}
	\item ¿Qué componentes básicos tiene una computadora?
	\item ¿Qué es un sistema operativo?
	\item ¿Qué tipos de archivos hay?
\end{itemize}

\subsection{Sistema Operativo GNU/Linux}
\begin{itemize}
	\item ¿Qué es Linux?
	\item Instalación de Linux Mint
	\item Como usar la terminal -- Comandos básicos y su funcionamiento.
	\item Uso de gparted, ls, cd, pwd y less
	\item Uso de argumentos en los comandos
	\item Como usar el gestor de paquetes apt.
\end{itemize}

\thispagestyle{fancy}
\subsection{Fundamentos de la Programación en Python}
\begin{itemize}
	\item Variables.
	\item Operadores.
	\item Condicionales.
	\item Listas, tuplas y diccionarios.
	\item Bucles.
	\item Funciones.
	\item Generadores
	\item Excepciones
	\item Programación orientada a Objetos
	      \begin{itemize}
		      \item Bases
		      \item Herencia
		      \item Polimorfismo
	      \end{itemize}
	\item Interfaces gráficas
\end{itemize}

\subsection{Uso de Git}
\begin{itemize}
	\item Bases de git
	\item Inicialización de un Repositorio.
	\item Uso del sistema de Commits.
	\item Creación de branches.
\end{itemize}

\subsection{Creación de documentos con \LaTeX}
\begin{itemize}
	\item Bases
	\item Práctica: Hacer un Curriculum Vitae
	\item Práctica: Creación de un Trabajo Práctico
\end{itemize}

\end{document}